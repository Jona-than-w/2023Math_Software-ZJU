\documentclass[UTF8]{ctexart}
\usepackage{setspace}
\usepackage[letterpaper,top=2cm,bottom=2cm,left=3cm,right=3cm,marginparwidth=1.75cm]{geometry}
\usepackage{listings}
\usepackage{xcolor}      %代码着色宏包

\usepackage{amsmath}
\usepackage{amssymb}\usepackage{mathabx}
\usepackage{amsfonts}
\usepackage{tikz}
\usepackage{verbatim}
\usepackage[mathscr]{eucal}
\usepackage{graphicx}
\usepackage{caption}
\usepackage{subfigure}
\CTEXsetup[format={\Large\bfseries}]{section}

\title{数学软件项目作业}
\author{3200103029 wangjinchen}
\date{July 2023}

\begin{document}

\maketitle
\section{工作内容}
step-3主要工作内容是求解
\begin{align*}
  -\Delta u &= f \qquad\qquad & \text{in}\ \Omega,
  \\
  u &= 0 \qquad\qquad & \text{on}\ \partial\Omega.
\end{align*}
其中$\Omega=[-1,1]^2$,$f(\mathbf x)=1$.\par
如果$\varphi$ 是$H^1$ 中的边值为零的测试函数,通过对以下公式分部积分
\begin{align*}
  -\int_\Omega \varphi \Delta u = \int_\Omega \varphi f.
\end{align*}
得到\begin{align*}
  (\nabla\varphi, \nabla u)
   = (\varphi, f),
\end{align*}
.问题是寻找$u$ 使得上式对所有符合要求的$\varphi$ 都成立.\par
通过step-1,step-2,我们能够实现三角形划分和形函数定义,分解函数形式为:$u_h(\mathbf x)=\sum_j U_j \varphi_j(\mathbf
x)$.则希望\begin{align*}
  (\nabla\varphi_i, \nabla u_h)
   = (\varphi_i, f),
   \qquad\qquad
   i=0\ldots N-1.
\end{align*}
故
\begin{align*}
  (\nabla\varphi_i, \nabla u_h)
  &= \left(\nabla\varphi_i, \nabla \Bigl[\sum_j U_j \varphi_j\Bigr]\right)
\\
  &= \sum_j \left(\nabla\varphi_i, \nabla \left[U_j \varphi_j\right]\right)
\\
  &= \sum_j \left(\nabla\varphi_i, \nabla \varphi_j \right) U_j.
\end{align*}

故写作矩阵形式是\begin{align*}
  A U = F,
\end{align*}
其中\begin{align*}
  A_{ij} &= (\nabla\varphi_i, \nabla \varphi_j),
  \\
  F_i &= (\varphi_i, f).
\end{align*}
为了求得积分,使用权数插值,即记\begin{align*}
    A_{ij} &= (\nabla\varphi_i, \nabla \varphi_j)
    = \sum_{K \in {\mathbb T}} \int_K \nabla\varphi_i \cdot \nabla \varphi_j,
    \\
    F_i &= (\varphi_i, f)
    = \sum_{K \in {\mathbb T}} \int_K \varphi_i f,
  \end{align*}
  我们使用\begin{align*}
    A^K_{ij} &=
    \int_K \nabla\varphi_i \cdot \nabla \varphi_j
    \approx
    \sum_q \nabla\varphi_i(\mathbf x^K_q) \cdot \nabla
    \varphi_j(\mathbf x^K_q) w_q^K,
    \\
    F^K_i &=
    \int_K \varphi_i f
    \approx
    \sum_q \varphi_i(\mathbf x^K_q) f(\mathbf x^K_q) w^K_q,
  \end{align*}
  来估计.\par
  下面介绍线性方程组的解法,
高斯消去法的复杂度是$O(N^3) $ ,过大.我们使用共轭梯度法,可以迭代收敛我们的解,复杂度是$O(N^2)$.\newpage
\section{程序内容}
程序求解的问题是 
\begin{align*}
  -\Delta u &= 1  & \text{in}\ \Omega,
  \\
  u &= 0  & \text{on}\ \partial\Omega.
\end{align*}
其中$\Omega=[-1,1]^2$.
求解结果如图
\begin{figure}[h]
    \centering
    \includegraphics[scale=0.5]{output.eps}
    \caption{温度图}
    \label{fig:enter-label}
\end{figure}


\end{document}
